\section{Organisation}

Cette section décrit la manière dont nous nous sommes organisés pour mener à bien ce projet.

\subsection{Répartition des tâches}

\begin{tabular}{|l|p{7cm}|l|}
\hline

Piller Bryan & 
\begin{itemize}
  \item Parties architecture et conception du projet
  \item Création des beans
\end{itemize}\\
\hline

Mathieu Clément & 
\begin{itemize}
  \item Implémentation du backend
  \item Configuration Maven   
\end{itemize}\\
\hline

Les deux &
\begin{itemize}
  \item Implémentation du frontend
  \item Rédaction du rapport
\end{itemize}\\
\hline

\end{tabular}

\subsection{Environnement de développement}
\begin{itemize}
\item IntelliJ IDEA
\item Eclipse
\item Git
\item GNU/Linux
\item Tomcat
\item Hibernate H2 in-memory database
\item Chromium Developer Tools
\item Extension de navigateur Postman (REST Client)
\end{itemize}

\subsection{Déploiement de \guillemets{production}}

Vous pouvez trouver respectivement le web service et le frontend aux adresses suivantes :

\url{http://langid.tic.eia-fr.ch:8080/pizzaorders-ws/rest/application.wadl}

\url{http://langid.tic.eia-fr.ch:8080/pizzaorders-frontend}

Note: par manque de temps nous avons préféré nous concentrer sur la partie backend, c'est pourquoi 
le frontend est si simple et son design est si ravageur.

\subsection{Maven}

% Compilation, dépendances, déploiement, properties




