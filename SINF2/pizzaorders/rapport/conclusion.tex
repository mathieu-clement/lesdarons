
\section*{Conclusion}
Ce travail nous a beaucoup plus. La liberté de choisir le thème et l'absence de restrictions sur les technologies 
à employer
ont fait de ce travail pratique une expérience agréable. Il est très formateur de pouvoir faire soi-même des choix, 
bons ou mauvais, qu'il s'agit ensuite de devoir défendre et assumer.

Concernant le travail en lui-même, on aurait souhaité passer plus de temps sur le frontend. Malheureusement 
en raison du rendu de multiples autres travaux et des examens ces dernières semaines, et en ayant utilisé tout 
le temps accordé sur les heures de cours et bien plus en dehors, quelques heures de plus nous ont tout de même manqué.

Nous nous sommes particulièrement intéressés à la partie serveur et web service. 
Deux raisons à cela : la première est que nous avons fait le choix d'avoir quelque chose de solide en arrière-plan car c'est souvent la source de beaucoup d'erreurs et
c'est là que le risque (intégrité des données, consistance, aspects légaux) est le plus important ;
la deuxième raison est que nous avons déjà par le passé créé des interfaces web mais nous manquons
d'expérience dans la gestion d'un backend, plus complexe et en ce qui nous concerne, plus intéressant à faire
que le choix --- aussi cornélien soit-il --- de la couleur d'un bouton.

Si vous le permettez, nous souhaiterions émettre la proposition suivante : il pourrait être plus judicieux de débuter le mini-projet
plus tôt dans le semestre. En particulier, le TP 11 est finalement très similaire à celui-ci excepté le fait que cette
fois-ci le thème était laissé libre.

D'avance, nous nous réjouissons de continuer l'exploration des frameworks et de mettre en pratique 
dans le futur 
les connaissances acquises dans le cadre de ce cours. Au terme de ce semestre, nous vous 
remercions pour vos conseils avisés et vous adressons nos meilleurs sentiments.
