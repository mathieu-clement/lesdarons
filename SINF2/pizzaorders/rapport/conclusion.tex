
\section*{Conclusion}
Nous avons aimé faire ce travail. La liberté de choisir le thème et l'absence de restrictions sur les technologies 
fait de cette expérience une chose agréable. Il est très formateur de pouvoir faire soi-même les choix, 
bons ou mauvais parfois, qu'il faut ensuite défendre et assumer.

Concernant le travail en lui-même, on aurait souhaité passer plus de temps sur le frontend. Malheureusement 
en raison du rendu de multiples autres travaux et des examens ces dernières semaines, et en ayant passé tout 
le temps accordé sur les heures de cours et bon multiple en dehors, le temps nous a quand même manqué.

Nous nous sommes particulièrement intéressé au côté serveur. Il y a deux raisons à cela : la première est que nous avons fait le choix d'avoir quelque chose de solide en arrière-plan car c'est souvent la source de beaucoup d'erreurs et
c'est là que le risque (intégrité des données, consistances, aspects légaux) est le plus important. 
La deuxième raison est que nous avons déjà par le passé créé des interfaces web mais que nous manquons
d'expérience dans la gestion d'un backend, plus complexe et en ce qui nous concerne, plus intéressant à faire
que le choix cornélien de la couleur d'un bouton.

Nous nous permettons toutefois une petite suggestion: il pourrait être plus judicieux de débuter le projet
plus tôt dans le semestre. En particulier, le TP 11 est finalement très similaire à celui-ci excepté le fait
que le thème était imposé la fois dernière.

Nous nous réjouissons de continuer l'exploration des frameworks et de mettre en pratique les connaissances
acquises dans le cadre de ce cours. Au terme de ce semestre, nous vous adressons nos bons sentiments.
