\subsection{Maven}

% Compilation, dépendances, déploiement, properties

Apache Maven s'est avéré pratique pour ce projet.
En plus de gérer impeccablement les dépendances, il nous a aussi
permis de déployer le WAR sur Tomcat.

\subsubsection{Téléchargement des sources}

Il est quasiment indispensable de posséder au moins la documentation
des APIs lorsque l'on utilise des bibliothèques et composants tiers.

Les sources peuvent aussi s'avérer utiles dans certains cas, 
lorsque la documentation est lacunaire ou pour comprendre
d'où proviennent certains bugs et messages d'erreur.

Les IDE comme IntelliJ IDEA vont chercher les sources au même endroit
qu'ils trouvent les artéfacts. On peut demander à Maven de télécharger
les sources au moyen de la commande 

\verb|mvn dependency:source|

\subsubsection{Déploiement sur Tomcat}

Tomcat dispose d'un \guillemets{accès script} (en fait via JMX) que l'on peut utiliser
pour déployer des web applications. L'autre solution consiste à
copier le WAR (ou son contenu, ce qui est par exemple dénommé \emph{exploded WAR} dans IntelliJ IDEA) 
dans le dossier \verb|webapps| surveillé par Tomcat,
un mécanisme commun à de nombreux \emph{servlet containers.}

Voici la configuration à ajouter dans le \verb|pom.xml|:

\begin{xmlcode}
<project>
    <build>
        <plugins>
            <plugin>
                <groupId>org.apache.tomcat.maven</groupId>
                <artifactId>tomcat7-maven-plugin</artifactId>
                <version>2.2</version>
                <executions>
                    <!-- Run before integration tests -->
                    <execution>
                        <phase>pre-integration-test</phase>
                        <goals>
                            <goal>redeploy</goal>
                        </goals>
                    </execution>
                </executions>
                <configuration>
                    <path>/pizzaorders-ws</path>
                    <update>true</update>
                    <url>http://localhost:8080/manager/text</url>
                    <!--
                    In Tomcat7 installation directory, add this to conf/tomcat-users.xml:

                    <role rolename="manager-script"/>
                    <user username="my-script-username" 
                          password="my-secret" 
                             roles="manager-script"/>

                    NEVER give both manager-script and manager-gui roles 
                    as the text and JMX interfaces
                    are not protected against CSRF!

                    -->
                    <username>langid-script</username>
                    <!-- or use tomcat.username property -->
                    <password>secret-langid</password>
                    <!-- or use tomcat.password property -->
                </configuration>
            </plugin>
        </plugins>
    </build>
</project>
\end{xmlcode}

Voir le commentaire XML concernant la configuration à ajouter à Tomcat et les précautions à prendre.

Ici nous avons indiqué vouloir que l'application soit déployée dans tous les cas
afin de pouvoir exécuter les tests d'intégration. Si on retire la section \verb|<executions>|
on peut alors demander le déploiement sur Tomcat avec \verb|mvn tomcat7:deploy|.

Notez que bien que le plugin ait été développé à l'origine pour Tomcat 7.0, il fonctionne toujours avec Tomcat 8.0.

\subsubsection{Propriétés}

Il est courant de devoir indiquer une information à de multiples reprises dans le \verb|pom.xml|.
Pour éviter la répétition, il est de coutume d'utiliser une \emph{propriété.}

Exemple :

\begin{xmlcode}
<project>
    <properties>
        <jersey.version>2.15</jersey.version>
    </properties>

    <dependencies>
        <dependency>
            <groupId>org.glassfish.jersey.containers</groupId>
            <artifactId>jersey-container-servlet-core</artifactId>
            <version>${jersey.version}</version>
    </dependencies>
</project>
\end{xmlcode}
