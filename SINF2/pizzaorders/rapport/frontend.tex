\section{Réalisation du frontend}
L'interface graphique devra permettre de choisir des pizzas et de passer commande au serveur.

\subsection{Technologies utilisées}

Pour réalisé une interface, nous avons choisi d'utiliser principalement AngularJS couplé avec les technologies standards du web.

\subsection{Incompatibilité avec le backend}
L'interface utilisateur n'est pas encore fini. Il reste des points d'amélioration comme le visuel ou encore quelques fonctionnalités présentes sur le serveur comme l'annulation de commande ou l'ajout de pizza.

On a aussi eu des changements à faire au niveau du code pour être compatible avec le backend.

\subsubsection{Changer les defaults}

Le but est de changer le type de contenu du header que l'on va transmettre au serveur pour être compatible, voici le code à insérer : 

\begin{javascriptcode}
$http.defaults.headers.post["Content-Type"] = "application/x-www-form-urlencoded";
\end{javascriptcode}

Maintenant il faut encore rendre les requête sous le bon format avec ce code :
\begin{javascriptcode}
 $http.defaults.transformRequest = function(obj) {
        var str = [];
        for(var p in obj)
            str.push(encodeURIComponent(p) + "=" + encodeURIComponent(obj[p]));
        return str.join("&");
    };
\end{javascriptcode}






