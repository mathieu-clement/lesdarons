\section{Réalisation du frontend}
L'interface graphique devra permettre de choisir des pizzas et de passer commande.

\subsection{Technologies utilisées}

Pour réaliser cette interface, nous avons choisi d'utiliser principalement AngularJS couplé à HTML5, CSS et consorts.

\subsection{Incompatibilité avec le backend}
L'interface utilisateur n'est pas terminée. Il reste des points d'amélioration comme le visuel ou encore quelques fonctionnalités présentes sur le serveur comme l'annulation de commande, l'ajout de pizzas et d'ingrédients.

Il y a également eu des changements à faire au niveau du code du frontend pour qu'il soit compatible avec le backend.

\subsubsection{Changer les par défaut}

Le but est de changer le type de contenu via les en-têtes transmis au serveur. Le code à insérer est le suivant :

\begin{javascriptcode}
$http.defaults.headers.post["Content-Type"] = "application/x-www-form-urlencoded";
\end{javascriptcode}

Les autres possibilités est le format \verb|form-data| et l'envoi des données sous forme brute. 
On peut modifier le \emph{Content-Type} pour chaque requête individuellement si nécessaire.

Ce changement n'est pas suffisant. AngularJS ne peut pas deviner qu'il doit transformer
le dictionnaire des \emph{data} sous la forme \guillemets{URL encodée}. Il faut le faire explicitement.
AngularJS nous offre fort heureusement un \emph{hook} pour le faire :

\begin{javascriptcode}
 $http.defaults.transformRequest = function(obj) {
        var str = [];
        for(var p in obj)
            str.push(encodeURIComponent(p) + "=" + encodeURIComponent(obj[p]));
        return str.join("&");
    };
\end{javascriptcode}

On critique souvent Javascript mais vous avouerez qu'il est difficile de faire plus court et plus clair.
