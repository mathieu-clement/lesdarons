
\section{Conception}

\subsection{But de l'application}

Cette application composée d'une partie frontend et backend (web service)
permet à des clients affamés de consulter la carte (avec nom des pizzas, liste des ingrédients
et prix) d'un service de livraison de pizzas
à domicile, de faire son choix et d'effectuer une commande après avoir entré une adresse de livraison.

Une interface permet ensuite aux employés de consulter la liste des commandes pour préparer
les pizzas et les faire livrer.

\subsection{Architecture générale / MVC}

\begin{tabular}{| l l p{11cm} |}
\hline
\bf Vue & HTML, CSS, AngularJS & La vue est constituée d'une page HTML unique mais permettant de passer
                                 de la carte des pizzas au formulaire de commande. \\
\hline
\bf Contrôleur & AngularJS, Jersey & L'aspect gestion du panier du client et une partie de la validation
                                     est effectuée côté navigateur, tandis que la mise à jour de la base
                                     de données et une autre partie de la validation est réalisée
                                     du côté du web service. \\
\hline
\bf Modèle & Entités Hibernate, BDD & Le modèle est géré grâce à des entités utilisant les annotations JPA.
                                      Les requêtes à la base de données et l'implémentation de l'ORM
                                      fait appel à Hibernate. \\
    \hline
\end{tabular}

% AngularJS (single page) / Web Service / Hibernate / DB

