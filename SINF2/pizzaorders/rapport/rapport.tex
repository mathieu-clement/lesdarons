%
% IMPORTANT:
% Document must be compiled with -shell-escape
%
\documentclass[twoside]{eiaArticle} 
%~ graphics path
%\graphicspath{resources/}

% ------------------------------------------------------------------------------
% Metadata
% ------------------------------------------------------------------------------
\usepackage{eiautils}
\usepackage{cleveref}
\usepackage{fancyvrb,cprotect} % cprotect for Verb in \*section{} command
\usepackage{emptypage}

% Faire apparaître les notes de bas de page insérées dans un tableau
\usepackage{footnote}
\makesavenoteenv{tabular}

\usepackage{wasysym}

\usepackage{babel}

\usepackage{textpos}

% basé sur s09

\title{Mini-projet }
\date{19 janvier 2015}

\eiasetupmeta[
    course=Systèmes d'information 2,  % also change after \begin{document} !!!
    tpno=S10, % also change after \begin{document} !!!
    author=Mathieu Cl\'ement, 
    authortwo=Bryan Piller, 
    class=TIC / I-3,
    prof=Omar {Abou Khaled} Stefano Carrino Joël Dumoulin,
    date=\thedate
]{}

\author{ \eiaauthors }
\eiasetuphdr

\usepackage{minted}
\usemintedstyle{colorful} %colorful, pastie, friendly are nice
\newminted{prolog}{fontsize=\footnotesize}
\newminted{java}{fontsize=\footnotesize}
\newminted{xml}{fontsize=\footnotesize}
\newminted{sql}{fontsize=\footnotesize}
\newminted{javascript}{fontsize=\footnotesize}

\newcommand\bellecitation[1]{\qquad\qquad \makebox{\emph{\guillemets{#1}}}}

\newcommand\bigsmile{{\LARGE\smiley{}}}

\usepackage{amssymb}% http://ctan.org/pkg/amssymb
\usepackage{pifont}% http://ctan.org/pkg/pifont
\newcommand{\cmark}{\ding{51}}%
\newcommand{\xmark}{\ding{55}}%

\newenvironment{absolutelynopagebreak}
  {\par\nobreak\vfil\penalty0\vfilneg
   \vtop\bgroup}
  {\par\xdef\tpd{\the\prevdepth}\egroup
   \prevdepth=\tpd}

\begin{document}
\chead{SINF2 -- TP12}
\eiafullpagetitlewithtocandribbon{} % use *fat* ribbon for very long titles

\cleardoublepage % First page of text must be on the right

\section*{Introduction}
\setcounter{page}{1}
Dans ce document, nous présentons le mini-projet créer durant le cours de système d'information 2. Le but était de créer une applications qui met en œuvre des technologies vue durant les séance de cours.
Nous avons décidé d'implémenter un système de commande de pizza, les idées sont pour la plupart basées sur des site connus comme \textit{Dominos pizza} pour ne citer que lui. Les fonctionnalités sont l'ajoute de pizza comme la "Quattro Formaggi" dans un panier, choisir la quantité voulu ainsi que sa grandeur et de passer commande en ajoutant les informations indispensable à une futur livraison, par exemple le nom de la rue.

Pour réalisé cette application nous avons utilisé ces technologies :
\begin{itemize}
\item Hibernate
\item AngularJS
\item JSON
\item Tomcat
\item REST
\end{itemize}


\section{Conception}

\subsection{But de l'application}

\subsection{Architecture générale / MVC}

% AngularJS (single page) / Web Service / Hibernate / DB

\subsection{MVC}




\section{Réalisation du backend}

\subsection{Technologies utilisées}

\subsection{Web Service}

\cprotect\subsubsection{\verb|web.xml|}

\subsubsection{Chemins REST}

% header Location

\cprotect\subsubsection{\verb|PUT| et \verb|POST|}

\cprotect\subsubsection{\verb|DELETE|}

\subsubsection{Codes de statut HTTP}

\subsection{Accès aux données}

\subsubsection{Technologies utilisées}

% Hibernate + H2 + JPA annotations

% Montrer les requêtes dans le log de Tomcat

\subsubsection{Insertion des données initiales}

% Lecture du XML
% Montrer le XML

\cprotect\subsubsection{Création de la \verb|SessionFactory|}

% HibernateListener

\cprotect\subsubsection{\verb|SessionFacade|}

% ORMFacade

\subsubsection{Problèmes des clés composées}

% AssocEntity + AssocId

\subsubsection{Requêtes HQL}

\subsubsection{Gestion des transactions}

% OrderManager

\subsection{JSON}



\section{Réalisation du frontend}
L'interface graphique devra permettre de choisir des pizzas et de passer commande au serveur.

\subsection{Technologies utilisées}

Pour réalisé une interface, nous avons choisi d'utiliser principalement AngularJS couplé avec les technologies standards du web.

\subsection{Incompatibilité avec le backend}
L'interface utilisateur n'est pas encore fini. Il reste des points d'amélioration comme le visuel ou encore quelques fonctionnalités présentes sur le serveur comme l'annulation de commande ou l'ajout de pizza.

On a aussi eu des changements à faire au niveau du code pour être compatible avec le backend.

\subsubsection{Changer les defaults}

Le but est de changer le type de contenu du header que l'on va transmettre au serveur pour être compatible, voici le code à insérer : 

\begin{javascriptcode}
\$http.defaults.headers.post["Content-Type"] = "application/x-www-form-urlencoded";
\end{javascriptcode}

Maintenant il faut encore rendre les requête sous le bon format avec ce code :
\begin{javascriptcode}
 \$http.defaults.transformRequest = function(obj) {
        var str = [];
        for(var p in obj)
            str.push(encodeURIComponent(p) + "=" + encodeURIComponent(obj[p]));
        return str.join("&");
    };
\end{javascriptcode}








\section{Validation}

% captures d'écran de Postman
% assocId

\section{Déploiement et mise en production}

\subsection{Proxy Apache}

% Port 8080
% mettre ensemble des applis de différents langages

\subsection{Base de données PostgreSQL}

\subsubsection{Création de la base de données}

% Dans la réalité il faudrait créer un utilisateur avec des droits les plus limités possibles.

\subsubsection{Modifications apportées au web service}




\section{Organisation}

Cette section décrit la manière dont nous nous sommes organisés pour mener à bien ce projet.

\subsection{Répartition des tâches}

\begin{tabular}{|l|p{7cm}|l|}
\hline

Piller Bryan & 
\begin{itemize}
  \item Parties architecture et conception du projet
  \item Création des beans
\end{itemize}\\
\hline

Mathieu Clément & 
\begin{itemize}
  \item Implémentation du backend
  \item Configuration Maven   
\end{itemize}\\
\hline

Les deux &
\begin{itemize}
  \item Implémentation du frontend
  \item Rédaction du rapport
\end{itemize}\\
\hline

\end{tabular}

\subsection{Environnement de développement}
\begin{itemize}
\item IntelliJ IDEA
\item Eclipse
\item Git
\item GNU/Linux
\item Tomcat
\item Hibernate H2 in-memory database
\item Chromium Developer Tools
\item Extension de navigateur Postman (REST Client)
\end{itemize}

\subsection{Déploiement de \guillemets{production}}

Vous pouvez trouver respectivement le web service et le frontend aux adresses suivantes :

\url{http://langid.tic.eia-fr.ch:8080/pizzaorders-ws/rest/application.wadl}

\url{http://langid.tic.eia-fr.ch:8080/pizzaorders-frontend}

Note: par manque de temps nous avons préféré nous concentrer sur la partie backend, c'est pourquoi 
le frontend est si simple et son design est si ravageur.

\subsection{Maven}

% Compilation, dépendances, déploiement, properties






\section*{Conclusion}
Nous avons apprécié faire ce travail. La liberté de choisir son thème et de ne pas avoir de restriction sur les technologies fait de cette expérience, une chose agréable. On aime pouvoir faire des choix, bon ou mauvais parfois, mais au moins on dois les assumer et on ne peut pas dire que c'est inutile ou qu'on ne l'utiliserai pas.

Sur le travail en lui-même, on aurait aimé faire plus pour la partie front-end, malheureusement le temps pour nous est précieux ! Nous avons priorisé le côté serveur plutôt que l'interface client. Il y a deux raisons à cela, la première est que nous avons fait le choix d'avoir quelque chose de solide en arrière car c'est souvent la source de beaucoup d'erreurs. La deuxième raison est que nous savons créer des interfaces intuitif et que c'est relativement facile à contrario, faire une gestion dite "back-end" est beaucoup plus complexe et nous avons priorisé notre temps dessus.

Une petite remarque, nous trouvons dommage d'attendre la fin du semestre pour mettre un mini-projet, l'expérience aurait été meilleur si elle avait eu lieu avant.

Sinon nous somme satisfaisant de notre applications, elle répond à nos attentes en terme de fonctionnalités si on ne prend pas en compte l'interface.
\end{document}

